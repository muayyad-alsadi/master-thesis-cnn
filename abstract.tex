\specialchap{Abstract}
{\fontsize{16pt}{24pt}\selectfont
\begin{center}\bfseries
Using Deep Convolutional Neural Networks and Knowledge Transfer for Image Recognition

By\\
Muayyad Alsadi

Supervisor\\
Prof. Arafat Awajan

Abstract
\end{center}
}

Image recognition is a topic of ``Computer Vision''
that aims to find and identify one or several specified objects, object classes, features or activities
in a given input image or video frame, even if it's partially obstructed from view.
Deep Convolutional Neural Networks is the-state-of-the-art technique for Image recognition,
but it requires a lot of training time and computing power to converge.

This thesis approaches the problem of image recognition with constrained budget in terms of
limited training time and limited computing power of typical commodity hardware.
Another imposed constraint is to have an expandable solution.

``Knowledge Transfer'' technique is used to reuse publicly available off-the-shelf trained models.
Several techniques are introduced to make it feasible, accelerate the process and make it expandable.

There are several applications for image recognition in different domains, such as
Face recognition, Optical Character Recognition, Manufacturing automatic inspection and Quality Control,
Medical diagnosis, and Autonomous vehicle related tasks such as Pedestrian detection.
The application discussed in this thesis is the task of identifying
brand, model and year of an image of a used car uploaded by a user of an e-Commerce service.
The importance of this application comes from the raising popularity of smart phones
equipped with cameras, where snapping a picture is far more convenient than typing description.

Trainable parameters (weights) are transferred from pre-trained ImageNet model into a model
that gets fine-tuned on a task of cleaning up noisy input dataset.
Then another transfer on a second model is used to identify most appropriate car models based on their market share,
and then it can be expanded to include more car models using a special proposed procedure.

Accuracy of more than 81\%  was achieved identifying 229 different car models,
by re-using off-the-shelf Inception V1 trained to solve ImageNet one thousand classes task having
accuracy of 69.8\% in that task.

To make sure the proposed method is generic and not domain specific,
several known academic tasks are also evaluated.

